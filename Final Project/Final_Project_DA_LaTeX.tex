\documentclass[paper=a4, fontsize=11pt]{scrartcl}

% ---------- Core ----------
\usepackage[margin=1in]{geometry}
\usepackage[T1]{fontenc}
\usepackage{lmodern}           % keep this; remove 'fourier' to avoid font conflicts
\usepackage[english]{babel}
\usepackage{microtype}         % nicer/wider line breaks; helps wrapping
\usepackage{amsmath,amsfonts,amsthm,amssymb}
\usepackage{graphicx}
\usepackage{float}
\usepackage{url}
\usepackage{placeins}  % gives \FloatBarrier


% ---------- Listings (R) ----------
\usepackage{listings}
\lstset{language=R,basicstyle=\ttfamily\small,breaklines=true}

% ---------- Section style & header/footer ----------
\usepackage{sectsty}
\allsectionsfont{\centering \normalfont\scshape}
\usepackage{fancyhdr}
\pagestyle{fancyplain}
\fancyhead{} \fancyfoot[L]{} \fancyfoot[C]{} \fancyfoot[R]{\thepage}
\renewcommand{\headrulewidth}{0pt}
\renewcommand{\footrulewidth}{0pt}
\setlength{\headheight}{13.6pt}

\numberwithin{equation}{section}
\numberwithin{figure}{section}
\numberwithin{table}{section}
\setlength\parindent{0pt}

% ---------- Tables with wrapping ----------
\usepackage{longtable,booktabs,array}
% Paragraph-style column types (wrapping). Adjust widths to taste; sum ≤ 1.00\textwidth
\newcolumntype{V}{>{\ttfamily\raggedright\arraybackslash}p{0.16\textwidth}} % Variable (mono)
\newcolumntype{D}{>{\raggedright\arraybackslash}p{0.52\textwidth}}          % Description
\newcolumntype{T}{>{\centering\arraybackslash}p{0.16\textwidth}}             % Type
\newcolumntype{A}{>{\centering\arraybackslash}p{0.14\textwidth}}             % Appendix
% If a longtable still nudges margins:
\setlength{\LTleft}{0pt}
\setlength{\LTright}{0pt}

% ---------- Hyperlinks last ----------
\usepackage{hyperref}
\hypersetup{hidelinks}

% ---------- Your title bits ----------
\newcommand{\horrule}[1]{\rule{\linewidth}{#1}}
\newcommand{\argmax}{\arg\!\max}
\title{
\normalfont \normalsize
\textsc{Mississippi State University} \\[25pt]
\horrule{0.5pt} \\[0.4cm]
\huge Data Analysis One\\
\horrule{2pt} \\[0.5cm]}

\author{Emily Hubbard}
\date{\normalsize\text{December 11, 2025}} % update if you like

\begin{document}


\maketitle % Print the title

%----------------------------------------------------------------------------------------
%	PROBLEM 1
%----------------------------------------------------------------------------------------

\section*{Final Report}
%\lipsum[2] % Dummy text
\textbf{1. Executive Summary}\\

\textbf{A. Introduction:}\\
Within the Bipolar I community, suicide is a real risk and safety concern. Compared to the general population, bipolar patients have a 20 times higher suicide risk, experiencing higher suicidal ideation, planning, and attempts (Izadi et al., 2023). Simultaneously, religion and spirituality can play a prominent role in reducing suicide risk within the public (Lawrence et al., 2016). However, some studies add complexity to this sentiment. For example, negative religious coping has been linked to worse outcomes in a sample of psychotic patients (Rosmarin et al., 2013).\\

So, despite the clinical significance of these research areas, there is little known about the overlap or relationship between religiosity and suicidality components (ideation, planning, and attempts) within the Bipolar I community.\\

With this study, my objective was to test whether a standardized religiosity index that combined importance of religion, prayer frequency, self-rated religiosity, and service attendance was associated with lifetime suicidality within the NSAL Bipolar I subgroup. In addition, I tested models that adjusted for sex, age, and the combination of the two. If a significant association is found, treatment options and recommended self-care practices could be informed and improved. If there is no association, those providing treatment can assess suicidal and religious struggles directly and separately if need be.\\

Our final analysis set includes BP-I respondents (N=40) with suicidality and all four religiosity items and 22 event occurrences. With a relatively small N and event occurences, it is important to be cautious when creating models. To keep the model reliable, I implemented a bias-reduced logistic regression (Firth), combined four religiosity items into a single standardized index, and limited covariates to a small set (age and sex) to avoid overfitting.\\

\textbf{B. Data}\\
The flow chart summarizes the data source, step-wise inclusion filters, and the final analytic cohort: CPES merged file → NSAL → lifetime Bipolar I → respondents with both suicidality and religiosity items.

\begin{figure}[h]
  \centering
  \includegraphics[width=.95\linewidth]{DA Final Project/Data_Flow_Chart.png}
  \caption{Data Flow Chart}
  \label{fig:Flowchart}
\end{figure}

\FloatBarrier
The table below lists all variables with short descriptions, measurement type, and links to the appendix for more information.

\section*{CPES\,/\,NSAL Bipolar \& Suicidality: Variable Cheat Sheet}

\vspace{0.25em}
\noindent\textbf{Cohort / IDs}
\begin{longtable}{V D}
\toprule
\textbf{Variable} & \textbf{Description} \\ \midrule
CPESPROJ & CPES component study (1 = NCS-R; 2 = NLAAS; \textbf{3 = NSAL}). Filter to CPESPROJ == 3. \\
CPESCASE & Respondent ID (unique within CPES merged file). \\
\bottomrule
\end{longtable}

\vspace{0.5em}
\noindent\textbf{Bipolar Filter}
\begin{longtable}{V D T A}
\toprule
\textbf{Variable} & \textbf{Description} & \textbf{Type} & \textbf{Appendix} \\ \midrule
\endfirsthead
\multicolumn{4}{l}{\small\itshape (continued)}\\
\toprule
\textbf{Variable} & \textbf{Description} & \textbf{Type} & \textbf{Appendix} \\ \midrule
\endhead
\bottomrule
\endfoot
V07842 & \textbf{Lifetime Bipolar I}. 1 = endorsed; 5 = not endorsed (special codes $\to$ NA). Keep V07842 == 1. &
Binary (Yes/No) & \textit{Figure A.1} \\
\end{longtable}

\vspace{0.5em}
\noindent\textbf{Lifetime Suicidality}
\begin{longtable}{V D T A}
\toprule
\textbf{Variable} & \textbf{Description} & \textbf{Type} & \textbf{Appendix} \\ \midrule
\endfirsthead
\multicolumn{4}{l}{\small\itshape (continued)}\\
\toprule
\textbf{Variable} & \textbf{Description} & \textbf{Type} & \textbf{Appendix} \\ \midrule
\endhead
\bottomrule
\endfoot
V00876 & Severe depression episode — \textbf{ever} thought about suicide (1 = Yes; 5 = No; specials $\to$ NA). &
Binary (Yes/No) & \textit{Figure A.2} \\
V00880 & Severe depression episode — \textbf{ever} made a suicide plan (1 = Yes; 5 = No). &
Binary (Yes/No) & \textit{Figure A.3} \\
V00882 & Severe depression episode — \textbf{ever} attempted suicide (1 = Yes; 5 = No). &
Binary (Yes/No) & \textit{Figure A.4} \\
V02044 & \textbf{Ever} attempted suicide (lifetime; outside MDE block) (1 = Yes; 5 = No). &
Binary (Yes/No) & \textit{Figure A.5} \\
\end{longtable}

\vspace{0.5em}
\noindent\textbf{Religiosity Predictors}
\begin{longtable}{V D T A}
\toprule
\textbf{Variable} & \textbf{Description} & \textbf{Type} & \textbf{Appendix} \\ \midrule
\endfirsthead
\multicolumn{4}{l}{\small\itshape (continued)}\\
\toprule
\textbf{Variable} & \textbf{Description} & \textbf{Type} & \textbf{Appendix} \\ \midrule
\endhead
\bottomrule
\endfoot
V06618 & Importance of religion in your life (1 = Very; 2 = Fairly; 3 = Not too; 4 = Not at all). Collapsed to High vs Low. &
Ordinal (4)\,$\rightarrow$\,Binary & \textit{Figure A.6} \\
V06614 & How often do you pray (1 = Nearly everyday; 2 = At least once a week; 3 = Few times a month; 4 = At least once a month; 5 = Few times a year; 6 = Never). Collapsed to Weekly+ / Yearly–Monthly / Never. &
Ordinal (6)\,$\rightarrow$\,3-level & \textit{Figure A.7} \\
V06621 & How religious are you (1 = Very; 2 = Fairly; 3 = Not too; 4 = Not at all). Used as ordinal (or collapsed if needed). &
Ordinal (4) & \textit{Figure A.8} \\
V06593 & Ever attended church services since age 18 (1 = Yes; 5 = No). &
Binary (Yes/No) & \textit{Figure A.9} \\

\texttt{relig\_index} & \textbf{Composite index (0–1).} Mean of available scored items (require $\ge$3 answered); higher$=$more religious. &
Continuous (0–1) &
\textit{} \\

\texttt{relig\_z} & \textbf{Standardized index.} z-score of \texttt{relig\_index} computed within the analysis set. &
Continuous (z) &
\textit{} \\
\end{longtable}

\vspace{0.5em}
\noindent\textbf{Covariates: Age \& Sex}
\begin{longtable}{V D T A}
\toprule
\textbf{Variable} & \textbf{Description} & \textbf{Type} & \textbf{Appendix} \\ \midrule
\endfirsthead
\multicolumn{4}{l}{\small\itshape (continued)}\\
\toprule
\textbf{Variable} & \textbf{Description} & \textbf{Type} & \textbf{Appendix} \\ \midrule
\endhead
\bottomrule
\endfoot

V07306 & Respondent \textbf{age in years} at interview. CPES special missings ($-9,-8,-7,97,98,99$) set to NA. In analysis we standardize to \texttt{age\_z}. &
Continuous (years) $\rightarrow$ z-score &
\textit{Figure A.10} \\

V09036 & Respondent \textbf{sex} (typically 1=Male, 2=Female). In analysis we recode to \texttt{sex\_male} (Male=1, Female=0). &
Binary (Male/Female) &
\textit{Figure A.11} \\

\end{longtable}







\textbf{C. Summary Information}\\
Preliminary results show that, in NSAL participants with lifetime Bipolar I who had both religiosity and suicidality measured (N=40; events=22), religiosity was not associated with lifetime suicidality. This was made evident using models with Firth correction as well as standard MLE (Maximum Likelihood Estimation) procedures. Even after adjusting for age and sex, no association was found between the two.\\

Below is a table of results from all fitted models that will be covered in this report.

\begin{table}[H]\centering
\begin{tabular}{lccc}
\toprule
Model & Predictor & OR & 95\% CI \\
\midrule
Firth logit & Religiosity z & 1.14 & 0.62–2.11 \\
MLE logit & Religiosity z & 1.15 & 0.62–2.15 \\
Firth logit & Religiosity z + Age z & 1.16 & 0.62–2.24 \\
Firth logit & Religiosity z + Male & 1.03 & 0.52–2.03 \\
Firth logit & Religiosity z + Age z + Male & 1.00 & 0.51–1.98 \\
\bottomrule
\end{tabular}
\caption{Primary models (unweighted).}
\end{table}



\textbf{2. Data Analysis}\\
\textbf{A. Preliminary Visuals}\\
To start, I took a look at the raw counts of suicidality outcomes within the NSAL Bipolar I subgroup. 

\begin{figure}[h]
  \centering
  \includegraphics[width=.5\linewidth]{LTSC_Percents.pdf}
  \caption{Lifetime Suicidality Components within Bipolar I subgroup}
  \label{LS Count by RT}
\end{figure}

\FloatBarrier
This figure shows the breakdown of each variable that contributes to the "lifetime suicidality" outcome. Note that they are not mutually exclusive and respondents can choose more than one.



\begin{figure}[h]
  \centering
  \includegraphics[width=.5\linewidth]{Lifetime_Suic_by_ReligTert.pdf}
  \caption{Counts of Lifetime Suicidality by Religious Tertile}
  \label{LS Count by RT}
\end{figure}

\FloatBarrier
This figure shows event counts for any lifetime suicidality (ideation, planning, or attempt) split by religiosity tertiles. Tertiles are based on the standardized religiosity index (z-score). Higher Score = Higher Religiosity. ‘Low’ is the bottom third of the index ($\leq$ 33rd percentile), ‘Mid’ is the middle third (33rd–67th percentiles), and ‘High’ is the top third ($>$ 67th percentile) among the analysis sample. \\

Although this figure does show a patterned increase by tertile, it is important to remember this is a small sample and the trend could be explained by simple variation. As such, a figure like this is simply exploratory in nature.\\

\textbf{B. Model Fitting}\\
Now, to begin model construction. Using a Firth model and the standardized religiosity measure to predict lifetime suicidality outcome, the following results were produced.\\

\begin{figure}[h]
  \centering
  \includegraphics[width=.9\linewidth]{Firth_Relig_Only_Results.png}
  \caption{Firth Model Results}
  \label{Firth Model Results}
\end{figure}

\FloatBarrier
\paragraph{Model Formula:}
\[
\operatorname{logit}\!\left\{\Pr(\texttt{suic\_life\_any}=1 \mid \texttt{relig\_z})\right\}
= \beta_0 + \beta_1\,\texttt{relig\_z}
\]

With estimates \(\widehat{\beta}_0=0.188\) and \(\widehat{\beta}_1=0.131\),
\[
\widehat{\operatorname{logit}} = 0.19 + 0.13\cdot \texttt{relig\_z}
\]

Equivalently, the odds ratio for a one–standard deviation increase in religiosity is
\[
\exp(\widehat{\beta}_1)=\exp(0.131)=1.14 \ \text{(95\% CI: 0.62,\ 2.11)}.
\]

What this odds ratio means is that for a 1 unit increase in the religiosity index , there is a 14\% increase in the odds of lifetime suicidality. However, the confidence interval includes 1 which means the findings are insignificant. This can also be concluded from the large p-value (0.67) associated with the religiosity index.\\


Next, we will look at a standard MLE model to confirm our findings.

\begin{figure}[h]
  \centering
  \includegraphics[width=.9\linewidth]{RegMLE_Relig_Results.png}
  \caption{MLE Model Results}
  \label{MLE Model Results}
\end{figure}

\FloatBarrier
\paragraph{Model Formula:}
\[
\widehat{\operatorname{logit}} = 0.19 + 0.14\cdot \texttt{relig\_z}
\]

Equivalently, the odds ratio for a one–standard deviation increase in religiosity is
\[
\exp(\widehat{\beta}_1)=\exp(0.139)=1.15 \ \text{(95\% CI: 0.62,\ 2.15)}.
\]

As with the Firth model, the confidence interval and large p-value (0.66) indicate insignificant association between religiosity and lifetime suicidality.\\

As a final measure, we will look at models that adjust for age, sex, and the combination of the two.\\

Here are the results of those three models:

\begin{figure}[h]
  \centering
  \includegraphics[width=.8\linewidth]{Firth_AB&C_Results.png}
  \caption{Adjusted Firth Models}
  \label{Adjusted Firth Models}
\end{figure}

\FloatBarrier
\paragraph{Model Formulas:}
\begin{align}
\nonumber
&\textbf{A: } \widehat{\operatorname{logit}} = 0.19 + 0.15\cdot \texttt{relig\_z} - 0.25\cdot \texttt{age\_z} \\
\nonumber
&\textbf{B: }\widehat{\operatorname{logit}} = 0.66 - 0.03\cdot \texttt{relig\_z} - 1.13\cdot \texttt{sex\_male}\\
\nonumber
&\textbf{C: }\widehat{\operatorname{logit}} = 0.64 - 0.00\cdot \texttt{relig\_z} - 0.23\cdot \texttt{age\_z} - 1.08\cdot \texttt{sex\_male}
\nonumber
\end{align} 

\begin{table}[H]\centering
\begin{tabular}{lcccc}
\toprule
Model & Predictor & OR & 95\% CI & P-values \\
\midrule
Firth logit & Religiosity z + Age z & 1.16 & 0.62–2.24 & 0.62, 0.41\\
Firth logit & Religiosity z + Male & 1.03 & 0.52–2.03 & 0.94, 0.09\\
Firth logit & Religiosity z + Age z + Male & 1.00 & 0.51–1.98 & 0.99, 0.47, 0.10\\
\bottomrule
\end{tabular}
\caption{Primary models (unweighted).}
\end{table}

Adjusting for age and sex leaves us with the same conclusion, there is no significant relationship between religiosity and lifetime suicidality.\\

\textbf{C. Results}\\
In the NSAL Bipolar I subset with complete measures in religiosity and lifetime suicidality (N=40; events=22), the standardized religiosity index was not associated with lifetime suicidality: Firth logistic odds ratio per 1-standard deviation = 1.14 (95\% CI 0.62–2.11, p=0.67). Adjusted models produced similar results, adding age, sex, or both. Descriptively, in Figure 0.3, event rates were higher in the “higher religiosity” tertile, but the small N means the figure is not significantly reliable.\\

Within this particular dataset, higher religiosity did not increase or decrease lifetime suicidality among individuals with Bipolar I. In practice, this means clinicians should assess suicide risk directly without assuming protection or harm from religious leaning. \\

\textbf{D. Limitations}\\
Small analysis set (N=40) with 22 events; cross-sectional measurements (some lifetime, some current); composite religiosity index based on equal weights could potentially be inaccurate in capturing religiosity. 


\newpage

\textbf{\underline{APPENDIX}}\\
\appendix
\section{Variables Explained:} 
\begin{figure}[h]
  \centering
  \includegraphics[width=.9\linewidth]{DA Final Project/V07842_Lifetime_BiP.png}
  \caption{V07842 (Lifetime Bipolar I)}
  \label{fig:V07842}
\end{figure}

\begin{figure}[h]
  \centering
  \includegraphics[width=.9\linewidth]{DA Final Project/V00976_SuicThought.png}
  \caption{V00876 (Suicidal Thought)}
  \label{fig:V00876}
\end{figure}

\begin{figure}[h]
  \centering
  \includegraphics[width=.9\linewidth]{DA Final Project/V00880_SuicPlan.png}
  \caption{V00880 (Suicidal Plan)}
  \label{fig:V00880}
\end{figure}

\begin{figure}[h]
  \centering
  \includegraphics[width=.9\linewidth]{DA Final Project/V00882_SuicAttempt_DuringDep.png}
  \caption{V00882 (Suicidal Attempt During Depression)}
  \label{fig:V00882}
\end{figure}

\begin{figure}[h]
  \centering
  \includegraphics[width=.9\linewidth]{DA Final Project/V02044_SuicAttempt.png}
  \caption{V02044 (Suicidal Attempt)}
  \label{fig:V02044}
\end{figure}

\begin{figure}[h]
  \centering
  \includegraphics[width=.9\linewidth]{DA Final Project/V06618_Religious_Importance.png}
  \caption{V06618 (Religious Importance)}
  \label{fig:V06618}
\end{figure}

\begin{figure}[h]
  \centering
  \includegraphics[width=.9\linewidth]{DA Final Project/V06614_Prayer_Frequency.png}
  \caption{V06614 (Prayer Frequency)}
  \label{fig:V06614}
\end{figure}

\begin{figure}[h]
  \centering
  \includegraphics[width=.9\linewidth]{DA Final Project/V06621_How_Rel_RU.png}
  \caption{V06621 (Religious Self-Rating)}
  \label{fig:V06621}
\end{figure}

\begin{figure}[h]
  \centering
  \includegraphics[width=.9\linewidth]{DA Final Project/V06593_Church_Attendance.png}
  \caption{V06593 (Church Attendance)}
  \label{fig:V06593}
\end{figure}

\begin{figure}[h]
  \centering
  \includegraphics[width=.9\linewidth]{DA Final Project/V07306_Age.png}
  \caption{V07306 (Age)}
  \label{fig:V07306}
\end{figure}

\begin{figure}[h]
  \centering
  \includegraphics[width=.9\linewidth]{DA Final Project/V09036_Sex.png}
  \caption{V09036 (Sex)}
  \label{fig:V09036}
\end{figure}

% --- Appendix: Sources cited in the Introduction ---
\FloatBarrier
\section{References}
 \renewcommand{\refname}{} 
\begin{thebibliography}{3}\small

\bibitem[Izadi et al., 2023]{izadi2023}
Izadi, N., Mitchell, R. H. B., Giacobbe, P., Nestor, D., Steinberg, R., Sinyor, M., \& Schaffer, A. (2023).
Suicide assessment and prevention in bipolar disorder: How current evidence can inform clinical practice.
\textit{Focus}, 21(4), 380--388.
\href{https://doi.org/10.1176/appi.focus.20230011}{https://doi.org/10.1176/appi.focus.20230011}

\bibitem[Lawrence et al., 2016]{lawrence2016}
Lawrence, R. E., Oquendo, M. A., \& Stanley, B. (2016).
Religion and suicide risk: A systematic review.
\textit{Archives of Suicide Research}, 20(1), 1--21.
\href{https://doi.org/10.1080/13811118.2015.1004494}{https://doi.org/10.1080/13811118.2015.1004494}

\bibitem[Rosmarin et al., 2013]{rosmarin2013}
Rosmarin, D. H., Bigda\mbox{-}Peyton, J. S., Öngur, D., Pargament, K. I., \& Björgvinsson, T. (2013).
Religious coping among psychotic patients: Relevance to suicidality and treatment outcomes.
\textit{Psychiatry Research}, 210(1), 182--187.
\href{https://doi.org/10.1016/j.psychres.2013.03.023}{https://doi.org/10.1016/j.psychres.2013.03.023}

\end{thebibliography}


\clearpage
\FloatBarrier
\section{R Code}
    \begin{lstlisting}[language=R]
#### Load Packages ####
suppressPackageStartupMessages({
  library(haven)
  library(dplyr)
  library(stringr)
  library(tidyr)
  library(purrr)
})

#### Load Overall Data ####
path <- "20240-0001-Data.dta"   #Merged CPES file where NSAL-> CPESPROJ == 3
dat  <- read_dta(path)

# Sanity check that we're in the merged file
stopifnot("CPESPROJ" %in% names(dat))   # 1=NCS-R, 2=NLAAS, 3=NSAL

#Setting CPES special missing codes to NA
special_na <- function(x) replace(x, x %in% c(-9, -8, -7, 97, 98, 99), NA)

# Label from a haven-labelled column
get_label <- function(x) {
  lab <- attr(x, "label"); if (is.null(lab)) "" else as.character(lab)
}

#### Restrict to NSAL & Lifetime Bipolar One Respondents ####
#NSAL
nsal <- dat %>% filter(CPESPROJ == 3) #NSAL respondents only

#Bipolar One 
nsal <- nsal %>%
  mutate(V07842 = special_na(V07842),
         bpi_life = case_when(V07842 == 1 ~ 1L,
                              V07842 == 5 ~ 0L,
                              TRUE ~ NA_integer_)) %>%
  filter(bpi_life == 1L). # V07842: 1=lifetime BP-I, 5=not lifetime BP-I (others NA)

#Prints rows left: 61 
cat("Rows after NSAL + lifetime BP-I restriction:", nrow(nsal), "\n")

#### Search for Lifetime Suicidality Variables ####
#Var/label lookup within subset
lbl_tbl <- tibble::tibble(
  var   = names(nsal),
  label = vapply(nsal, get_label, FUN.VALUE = character(1))
)

#Inspect lifetime/ever suicidality candidates (by label search)
life_candidates <- lbl_tbl %>%
  mutate(label_low = str_to_lower(label)) %>%
  filter(
    str_detect(label_low, "suicid"),
    str_detect(label_low, "ever|lifetime|times attempted|ever attempted|ever thought|ever made|ever plan")
  ) %>%
  mutate(non_missing = map_int(var, ~ sum(!is.na(nsal[[.x]])))) %>%
  arrange(desc(non_missing))

#Prints all lifetime suicidality variables:
cat("\nLifetime suicidality candidates (sorted by coverage):\n")
print(life_candidates, n = 50)

#### Search for 12-month Suicidality Variables ####

#Find the 12-month suicidality items (by label search)
suic_candidates <- lbl_tbl %>%
  mutate(label_low = str_to_lower(label)) %>%
  filter(
    str_detect(label_low, "suicid") &
      (str_detect(label_low, "12") | str_detect(label_low, "past 12") |
         str_detect(label_low, "last 12") | str_detect(label_low, "12 m"))
  ) %>%
  mutate(non_missing = map_int(var, ~ sum(!is.na(nsal[[.x]])))) %>%
  arrange(desc(non_missing))

cat("\n--- Possible 12-month suicidality variables (merged file) ---\n")
print(suic_candidates, n = 50)

#Build the 12-month suicidality outcome inside NSAL + BP-I
suic12_vars <- c("V01995",  # thought in past 12 mths
                 "V01999",  # plan in past 12 mths
                 "V02004")  # attempt in past 12 mths

# clean specials and build component/combined outcomes
nsal <- nsal %>%
  mutate(
    across(all_of(suic12_vars),
           ~ replace(., . %in% c(-9, -8, 97, 98, 99), NA_integer_)),
    suic12_thought = as.integer(.data[[suic12_vars[1]]] == 1L),
    suic12_plan    = as.integer(.data[[suic12_vars[2]]] == 1L),
    suic12_attempt = as.integer(.data[[suic12_vars[3]]] == 1L)
  ) %>%
  mutate(
    any_suic12 = as.integer(
      (suic12_thought == 1L) | (suic12_plan == 1L) | (suic12_attempt == 1L)
    )
  )

cat("\nCoverage per item (non-missing):\n")
print(sapply(suic12_vars, function(v) sum(!is.na(nsal[[v]]))))

cat("\n12-month suicidality (any):\n").   
print(table(nsal$any_suic12, useNA = "ifany"))

cat("\nBreakdown by item (yes counts):\n")
print(colSums(nsal[, c("suic12_thought","suic12_plan","suic12_attempt")] == 1, na.rm = TRUE))

cat("\nExample rows with any 12m suicidality:\n")
nsal %>%
  filter(any_suic12 == 1) %>%
  select(CPESCASE, any_suic12, suic12_thought, suic12_plan, suic12_attempt) %>%
  head(10) %>% print()

#### Realized there were not enough 12-month suicidality outcomes to create a stable model. Fell back to lifetime variables with larger N.

#Lifetime suicidality composites
life_items <- c("V00876", "V00880", "V00882", "V02044")  # ideation, plan, attempt-in-MDE, ever attempt
life_items <- intersect(life_items, names(nsal))

nsal <- nsal %>%
  mutate(across(all_of(life_items),
                ~ replace(., . %in% c(-9, -8, 97, 98, 99), NA_integer_)))

#Recode components as 0/1 with NAs preserved
yes1 <- function(x) as.integer(x == 1L)

comp_mat <- nsal %>%
  transmute(
    suic_life_ideation   = if ("V00876" %in% life_items) yes1(.data[["V00876"]]) else NA_integer_,
    suic_life_plan       = if ("V00880" %in% life_items) yes1(.data[["V00880"]]) else NA_integer_,
    suic_life_attemptMDE = if ("V00882" %in% life_items) yes1(.data[["V00882"]]) else NA_integer_,
    suic_life_attemptEver= if ("V02044" %in% life_items) yes1(.data[["V02044"]]) else NA_integer_
  )

# any lifetime & attempt lifetime with proper NA when all components missing
any_life <- {
  n_obs <- rowSums(!is.na(comp_mat))
  has_any <- as.integer(rowSums(comp_mat, na.rm = TRUE) > 0)
  has_any[ n_obs == 0 ] <- NA_integer_
  has_any
}
attempt_life <- {
  att_mat <- comp_mat[, c("suic_life_attemptMDE","suic_life_attemptEver")]
  n_obs <- rowSums(!is.na(att_mat))
  has_any <- as.integer(rowSums(att_mat, na.rm = TRUE) > 0)
  has_any[ n_obs == 0 ] <- NA_integer_
  has_any
}

nsal <- bind_cols(nsal, comp_mat) %>%
  mutate(
    suic_life_any     = any_life,
    suic_life_attempt = attempt_life
  )

cat("\nCoverage (non-missing) per lifetime item:\n")
print(sapply(life_items, function(v) sum(!is.na(nsal[[v]]))))

cat("\nLifetime suicidality (any):\n")
print(table(nsal$suic_life_any, useNA = "ifany"))

cat("\nLifetime suicide attempt (composite):\n")
print(table(nsal$suic_life_attempt, useNA = "ifany"))

cat("\nBreakdown of 'any' by lifetime components (yes counts):\n")
print(colSums(comp_mat == 1, na.rm = TRUE))

#### Religiosity items & Coverage/Overlap ####
relig4 <- c("V06618","V06614","V06621","V06593")

cov_tbl <- tibble::tibble(
  var         = relig4,
  label       = map_chr(relig4, ~ {lab <- attr(nsal[[.x]], "label"); if (is.null(lab)) .x else as.character(lab)}),
  non_missing = map_int(relig4, ~ sum(!is.na(nsal[[.x]]))),
  pct         = round(non_missing / nrow(nsal) * 100, 1)
)
cat("\nReligiosity coverage in NSAL + BP-I subset:\n")
print(cov_tbl)

nsal <- nsal %>%
  mutate(
    relig_all4     = rowSums(across(all_of(relig4), ~ !is.na(.))) == 4,
    relig_atleast2 = rowSums(across(all_of(relig4), ~ !is.na(.))) >= 2
  )

cat("\nOverlap with outcomes (primary=lifetime any; secondary=lifetime attempt):\n")
print(with(nsal, table(relig_all4,     suic_life_any,     useNA = "ifany")))
print(with(nsal, table(relig_atleast2, suic_life_any,     useNA = "ifany")))
print(with(nsal, table(relig_all4,     suic_life_attempt, useNA = "ifany")))
print(with(nsal, table(relig_atleast2, suic_life_attempt, useNA = "ifany")))

cat("\nRows that have lifetime-any=1 AND all 4 relig items present (peek):\n")
nsal %>%
  filter(suic_life_any == 1, relig_all4) %>%
  transmute(
    CPESCASE,
    V06618 = haven::as_factor(V06618),
    V06614 = haven::as_factor(V06614),
    V06621 = haven::as_factor(V06621),
    V06593 = haven::as_factor(V06593)
  ) %>% print(n = Inf)


##### Flow Chart Visual ####
#Counts from your current objects
library(dplyr)

N_cpes   <- nrow(dat)
N_nsal   <- sum(dat$CPESPROJ == 3, na.rm = TRUE)
N_bpi    <- nrow(nsal)

nsal$suic_life_any <- ifelse(is.infinite(nsal$suic_life_any), NA, nsal$suic_life_any)

N_suic_nonmiss   <- sum(!is.na(nsal$suic_life_any))
N_suic_events    <- sum(nsal$suic_life_any == 1, na.rm = TRUE)
N_relig_all4     <- sum(nsal$relig_all4, na.rm = TRUE)
N_overlap_any    <- sum(nsal$relig_all4 & !is.na(nsal$suic_life_any), na.rm = TRUE)
N_overlap_events <- sum(nsal$relig_all4 & nsal$suic_life_any == 1, na.rm = TRUE)

#Flowchart
library(DiagrammeR); library(glue)

grViz(glue('
digraph flow {{
  graph [rankdir=LR, fontsize=10]
  node  [shape=box, style="rounded,filled", fillcolor="#EEF5FF"]
  edge  [color="#6b6b6b"]

  CPES [label="CPES (N={N_cpes})"]
  NSAL [label="NSAL (N={N_nsal})"]
  BPI  [label="Lifetime BP-I (N={N_bpi})"]
  SUIC [label="Lifetime suicidality measured\\n(N={N_suic_nonmiss}; events={N_suic_events})"]
  REL  [label="All 4 relig items present (N={N_relig_all4})"]
  OVER [label="Overlap (both measured)\\nN={N_overlap_any}\\nEvents in overlap={N_overlap_events}"]

  CPES -> NSAL -> BPI
  BPI  -> SUIC
  BPI  -> REL
  SUIC -> OVER
  REL  -> OVER
}}'))


#### Religiosity Index & Visuals ####
suppressPackageStartupMessages({
  library(ggplot2)
})

# Build a single religiosity index
# Mapping:
# V06618: 1=very, 2=fairly, 3=not too, 4=not at all  (High ->1, Low->0)
# V06614: 1=nearly every day, 2=≥1/wk, 3=few/m, 4=≥1/m, 5=few/yr, 6=never (more ->1)
# V06621: 1=very religious, 2=fairly, 3=not too, 4=not at all (more ->1)
# V06593: 1=yes attended since 18, 5=no (yes ->1)
nsal <- nsal %>%
  mutate(
    relig_imp_num = case_when(V06618 %in% c(1,2) ~ 1,
                              V06618 %in% c(3,4) ~ 0,
                              TRUE ~ NA_real_),
    pray_num      = case_when(V06614 %in% c(1,2) ~ 1,
                              V06614 %in% c(3,4,5) ~ 0.5,
                              V06614 == 6          ~ 0,
                              TRUE ~ NA_real_),
    self_rel_num  = case_when(V06621 == 1 ~ 1,
                              V06621 == 2 ~ 2/3,
                              V06621 == 3 ~ 1/3,
                              V06621 == 4 ~ 0,
                              TRUE ~ NA_real_),
    attend_num    = case_when(V06593 == 1 ~ 1,
                              V06593 == 5 ~ 0,
                              TRUE ~ NA_real_),
    relig_items_answered = rowSums(across(c(relig_imp_num, pray_num, self_rel_num, attend_num),
                                          ~ !is.na(.))),
    relig_index = ifelse(relig_items_answered >= 3,
                         rowMeans(across(c(relig_imp_num, pray_num, self_rel_num, attend_num)),
                                  na.rm = TRUE),
                         NA_real_),
    relig_z = as.numeric(scale(relig_index))
  )

#Analytic cohort:
anal <- nsal %>%
  filter(!is.na(suic_life_any), relig_all4, !is.na(relig_index)) %>%
  mutate(
    r_tertile = factor(
      dplyr::ntile(relig_index + runif(dplyr::n(), -1e-9, 1e-9), 3),
      labels = c("Low","Mid","High")
    )
  )


cat("\nAnalysis N (overlap):", nrow(anal),
    " | events:", sum(anal$suic_life_any == 1, na.rm = TRUE), "\n")

#Visuals:

#Component bar of lifetime suicidality (in the whole BP-I subset where measured)
comp_long <- nsal %>%
  filter(!is.na(suic_life_any)) %>%
  transmute(
    ideation = suic_life_ideation,
    plan     = suic_life_plan,
    attempt  = suic_life_attempt
  ) |>
  tidyr::pivot_longer(everything(), names_to = "component", values_to = "yes") |>
  dplyr::summarise(n_yes = sum(yes == 1, na.rm = TRUE), .by = component)

p1 <- ggplot(comp_long, aes(x = component, y = n_yes)) +
  geom_col() +
  labs(title = "Lifetime suicidality components (BP-I in NSAL)",
       x = NULL, y = "Yes count") +
  theme_minimal(base_size = 12)

print(p1)

library(dplyr)
library(tidyr)
library(ggplot2)
library(scales)

comp_long <- nsal %>%
  filter(!is.na(suic_life_any)) %>%               
  transmute(
    ideation = suic_life_ideation,
    plan     = suic_life_plan,
    attempt  = suic_life_attempt
  ) %>%
  pivot_longer(everything(), names_to = "component", values_to = "yes") %>%
  # summarize counts + denominators for each component
  summarise(
    n      = sum(!is.na(yes)),
    n_yes  = sum(yes == 1, na.rm = TRUE),
    prop   = n_yes / n,
    .by    = component
  ) %>%
  # nice labels and ordering
  mutate(
    component = factor(component,
                       levels = c("ideation","plan","attempt"),
                       labels = c("Ideation","Plan","Attempt")),
    label_counts   = as.character(n_yes),                                 # just counts
    label_full     = paste0(n_yes, "/", n, " (", percent(prop, 1), ")")  # counts + %
  )

#Visual:
p1_counts <- ggplot(comp_long, aes(x = component, y = n_yes)) +
  geom_col(width = 0.7) +
  geom_text(aes(label = label_counts), vjust = -0.3, size = 4) +
  scale_y_continuous(expand = expansion(mult = c(0, 0.15))) +
  labs(title = "Lifetime suicidality components (BP-I in NSAL)",
       x = NULL, y = "Yes count") +
  theme_minimal(base_size = 12)
print(p1_counts)

p1_full <- ggplot(comp_long, aes(x = component, y = n_yes)) +
  geom_col(width = 0.7) +
  geom_text(aes(label = label_full), vjust = -0.3, size = 3.6) +
  scale_y_continuous(expand = expansion(mult = c(0, 0.20))) +
  labs(title = "Lifetime suicidality components (BP-I in NSAL)",
       x = NULL, y = "Yes count") +
  theme_minimal(base_size = 12)
print(p1_full)

#Event rate by religiosity tertile (in analysis overlap N≈40)
rate_tbl <- anal |>
  dplyr::summarise(
    n      = dplyr::n(),
    events = sum(suic_life_any == 1, na.rm = TRUE),
    prop   = events / n,
    label  = paste0(events, "/", n),  
    .by    = r_tertile
  )

p2 <- ggplot(rate_tbl, aes(x = r_tertile, y = prop)) +
  geom_col() +
  # put the counts just above each bar:
  geom_text(aes(y = prop + 0.03, label = label), vjust = 0, size = 4) +
  scale_y_continuous(
    labels = scales::percent_format(accuracy = 1),
    limits = c(0, 1),                       
    expand = expansion(mult = c(0, 0.05))   
  ) +
  labs(
    title = "Lifetime suicidality (any) by religiosity tertile",
    x = "Religiosity", y = "Event rate"
  ) +
  theme_minimal(base_size = 12)

print(p2)



#### First Model: Firth Logistic & MLE for Comparison ####
#Helper to print ORs nicely
or_table <- function(fit, use_profile = FALSE) {
  cf <- coef(fit);  # vector
  SE <- sqrt(diag(vcov(fit)))
  if (!use_profile) {
    ci <- cbind(cf - 1.96*SE, cf + 1.96*SE)
  } else {
    ci <- confint(fit)  # slow, profile CI
  }
  tibble::tibble(
    term = names(cf),
    OR   = exp(cf),
    low  = exp(ci[,1]),
    high = exp(ci[,2])
  )
}


library(logistf)
#Primary predictor = relig_z (per +1 SD religiosity)
#Firth
if (requireNamespace("logistf", quietly = TRUE)) {
  fit_firth <- logistf::logistf(suic_life_any ~ relig_z, data = anal)
  cat("\nFirth logistic (penalized) on overlap:\n"); print(summary(fit_firth))
  firth_or <- tibble::tibble(
    term = names(fit_firth$coefficients),
    OR   = exp(fit_firth$coefficients),
    low  = exp(fit_firth$ci.lower),
    high = exp(fit_firth$ci.upper)
  )
  cat("\nFirth ORs (95% CI):\n"); print(firth_or)
} else {
  cat("\nPackage 'logistf' not installed → running plain MLE glm() only.\n")
}

#Plain MLE (for reference)
fit_mle <- glm(suic_life_any ~ relig_z, data = anal, family = binomial())
cat("\nMLE logistic on overlap:\n"); print(summary(fit_mle))
cat("\nMLE ORs (Wald 95% CI):\n"); print(or_table(fit_mle))


#### Running Model to include Age and Sex ####
#Clean covariates (Age = V07306, Sex = V09036) and add to analytic cohort
anal <- anal %>%
  mutate(
    age_raw  = special_na(V07306),
    age_num  = suppressWarnings(as.numeric(age_raw)),
    age_z    = as.numeric(scale(age_num)),           # z-score (mean 0, sd 1)
    sex_raw  = special_na(V09036),
    # CPES/NSAL convention is typically 1=Male, 2=Female
    sex_male = dplyr::case_when(sex_raw == 1L ~ 1L,
                                sex_raw == 2L ~ 0L,
                                TRUE          ~ NA_integer_)
  )

# Quick coverage check inside the overlap set you’ll model
cat("\nCoverage in overlap set:\n")
print(tibble::tibble(
  N_overlap = nrow(anal),
  N_events  = sum(anal$suic_life_any == 1, na.rm = TRUE),
  N_age     = sum(!is.na(anal$age_z)),
  N_sex     = sum(!is.na(anal$sex_male)),
  N_both    = sum(!is.na(anal$age_z) & !is.na(anal$sex_male))
))

#Helper to fit Firth and return a tidy OR table
or_table_firth <- function(fit) {
  ci <- suppressMessages(confint(fit))
  tibble::tibble(
    term = names(coef(fit)),
    OR   = exp(coef(fit)),
    low  = exp(ci[, 1]),
    high = exp(ci[, 2])
  )
}

#Fit Adjusted Firth models:

# (A) relig_z + age
dat_A <- anal %>% dplyr::filter(!is.na(suic_life_any), !is.na(relig_z), !is.na(age_z))
fit_A <- logistf::logistf(suic_life_any ~ relig_z + age_z, data = dat_A)
cat("\nFirth: suic_life_any ~ relig_z + age_z  (n=", nrow(dat_A),
    ", events=", sum(dat_A$suic_life_any==1), ")\n", sep = "")
print(or_table_firth(fit_A))

# (B) relig_z + sex
dat_B <- anal %>% dplyr::filter(!is.na(suic_life_any), !is.na(relig_z), !is.na(sex_male))
fit_B <- logistf::logistf(suic_life_any ~ relig_z + sex_male, data = dat_B)
cat("\nFirth: suic_life_any ~ relig_z + sex_male  (n=", nrow(dat_B),
    ", events=", sum(dat_B$suic_life_any==1), ")\n", sep = "")
print(or_table_firth(fit_B))

# (C) relig_z + age + sex
dat_C <- anal %>% dplyr::filter(!is.na(suic_life_any), !is.na(relig_z),
                                !is.na(age_z), !is.na(sex_male))
fit_C <- logistf::logistf(suic_life_any ~ relig_z + age_z + sex_male, data = dat_C)
cat("\nFirth: suic_life_any ~ relig_z + age_z + sex_male  (n=", nrow(dat_C),
    ", events=", sum(dat_C$suic_life_any==1), ")\n", sep = "")
print(or_table_firth(fit_C))

# (A) relig_z + age
summary(fit_A)
# (B) relig_z + sex
summary(fit_B)
# (C) relig_z + age + sex
summary(fit_C)
    \end{lstlisting}


\end{document}
